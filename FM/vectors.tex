% Created 2026-02-04 Wed 12:44
% Intended LaTeX compiler: pdflatex
\documentclass[11pt]{article}
\usepackage[utf8]{inputenc}
\usepackage[T1]{fontenc}
\usepackage{graphicx}
\usepackage{longtable}
\usepackage{wrapfig}
\usepackage{rotating}
\usepackage[normalem]{ulem}
\usepackage{amsmath}
\usepackage{amssymb}
\usepackage{capt-of}
\usepackage{hyperref}
\date{\today}
\title{}
\hypersetup{
 pdfauthor={},
 pdftitle={},
 pdfkeywords={},
 pdfsubject={},
 pdfcreator={Emacs 30.2 (Org mode 9.7.11)}, 
 pdflang={English}}
\begin{document}

\tableofcontents

\section{Direction Vectors}
\label{sec:org20362d0}

\(r = O + tD\)  Is the general equation of a ray. If you want to know if a point belongs on a line you can substitute it as \(r\) and find the appropiate value of \(t\).
So lets assume, \(\vec{O} = (2, 1, 1) \text{ and } \vec{D} = (1, 1, 1)\) and we want to know if the point \(\vec{r} = (3, 2, 2)\) lies on the line. First we substite
our value of \(r\) to get: \((3, 2, 2) = (2, 1, 1) + t(1, 1, 1)\), \(t = 1\) so our point lies on the line.
\subsection{Simplifying Direction Vectors}
\label{sec:org0368b64}
The magnitude of a direction vector doesnt matter only the direction does, so they are typically represented as unit vectors to keep our values of \(t\) clean.
\subsection{Parametric Form}
\label{sec:org9e9016f}
By letting \(r = (x, y ,z)\) we can create three seperate equations: \((x, y, z) = (a_1, a_2, a_3) + t(b_1, b_2, b_3)\) to give use three similatenous equations like so:
\(r_n = a_n + tb_n\)
\end{document}
